\include{settings}

\begin{document}

\begin{titlepage}
\begin{center}
	САНКТ-ПЕТЕРБУРГСКИЙ ПОЛИТЕХНИЧЕСКИЙ УНИВЕРСИТЕТ\\ ПЕТРА ВЕЛИКОГО\\[0.3cm]
	\par\noindent\rule{10cm}{0.4pt}\\[0.3cm]
	Институт компьютерных наук и технологий \\[0.3cm]
	Кафедра компьютерных систем и программных технологий\\[4cm]
	
	Отчет по лабораторной работе № 1\\[3mm]
	Дисциплина: <<Базы данных>>\\[3mm]
	Тема: <<Разработка структуры БД>>\\[7cm]
\end{center}

\begin{flushleft}
	\hspace*{5mm} Выполнил студент гр. 43501/3  \hspace*{2.5cm}\sign[3cm]\hspace*{3.0mm} А.Ю. Ламтев\\
	\hspace*{10.4cm} (подпись)\\[3mm]
	\hspace*{5mm} Преподаватель \hspace*{6.0cm}\sign[3cm]\hspace*{2mm} А.В. Мяснов\\
	\hspace*{10.4cm} (подпись)\\[3mm]
	\hspace*{11.1cm} <<\sign[7mm]>> \sign[27mm] \the\year\hspace{1mm} г.
\end{flushleft}

\vfill

\begin{center}
	Санкт-Петербург\\
	\the\year
\end{center}
\end{titlepage}
\addtocounter{page}{1}

\tableofcontents
\newpage

\section{Цели работы}

Познакомитьcя с языком создания запросов управления данными SQL-DML. 

\section{Программа работы}

\begin{enumerate}
	\item Изучение SQL-DML.
	\item Выполнение всех запросов из списка стандартных запросов. Демонстрация результатов преподавателю.
	\item Получение у преподавателя и реализация SQL-запросов в соответствии с индивидуальным заданием. 			\item Демонстрация результатов преподавателю.
	\item Сохранение в БД выполненных запросов \code{SELECT} в виде представлений, запросов \code{INSERT}, \code{UPDATE} или \code{DELETE} --- в виде ХП.
 
\end{enumerate}
 
\section{Стандартные запросы} 

\subsection{Выборка данных из одной таблицы с использованием логических операций, \code{LIKE}, \code{BETWEEN}, \code{IN}}

В листинге \ref{lst:like-between-in-1.sql} представлен запрос, формирующий выборку пользователей мужского пола, родившихся до 30 декабря 2001 года, и длина логина которых находится в диапазоне между 7 и 11.

\lstinputlisting[caption={like-between-in-1.sql},label={lst:like-between-in-1.sql}]{like-between-in-1.sql}
 
Выборка, сформированная данным запросом, представлена на рис. \ref{fig:like-between-in-1}.

\begin{figure}[H]
	\centering
	\includegraphics[width=1.0\textwidth]{like-between-in-1.png}
	\caption{Выборка, сформированная \code{like-between-in-1.sql}}
	\label{fig:like-between-in-1}
\end{figure}
 

В листинге \ref{lst:like-between-in-2.sql} представлен запрос, формирующий выборку пользователей женского пола, логин которых начинается с <<\code{Marie.}>>.

\lstinputlisting[caption={like-between-in-2.sql},label={lst:like-between-in-2.sql}]{like-between-in-2.sql}

Выборка, сформированная данным запросом, представлена на рис. \ref{fig:like-between-in-2}.

\begin{figure}[H]
	\centering
	\includegraphics[width=1.0\textwidth]{like-between-in-2.png}
	\caption{Выборка, сформированная \code{like-between-in-2.sql}}
	\label{fig:like-between-in-2}
\end{figure}

В листинге \ref{lst:like-between-in-3.sql} представлен запрос, формирующий выборку фильмов, цена которых \$$5$ или \$$7$, при этом они не являются эпизодами сериалов, они были выпущены в октябре (не важно какого года), и их \code{IMDB} рэйтинг лежит в диапазоне между $7.5$ и $7.6$. Поля выборки следующие: идентификатор, стоимость, возраст в годах и рейтинг \code{IMDB}.

\lstinputlisting[caption={like-between-in-3.sql},label={lst:like-between-in-3.sql}]{like-between-in-3.sql}

Выборка, сформированная данным запросом, представлена на рис. \ref{fig:like-between-in-3}.

\begin{figure}[H]
	\centering
	\includegraphics[width=0.5\textwidth]{like-between-in-3.png}
	\caption{Выборка, сформированная \code{like-between-in-3.sql}}
	\label{fig:like-between-in-3}
\end{figure}

\subsection{Запрос с вычисляемым полем}

В листинге \ref{lst:calculated-field.sql} представлен запрос, формирующий выборку из 5-ти еще не закончившихся, автоматически возобновляемых подписок, длительность которых равна 30 дням. Поля у выборки следующие: идентификатор подписки, идентификатор пользователя, стоимость и число дней до завершения подписки. При этом последнее поле является вычисляемым.

\lstinputlisting[caption={calculated-field.sql},label={lst:calculated-field.sql}]{calculated-field.sql}

Выборка, сформированная данным запросом, представлена на рис. \ref{fig:calculated-field}.

\begin{figure}[H]
	\centering
	\includegraphics[width=0.5\textwidth]{calculated-field.png}
	\caption{Выборка, сформированная \code{calculated-field.sql}}
	\label{fig:calculated-field}
\end{figure}

Значения в столбце \code{expires\_in\_days}, превосходящие 30, объясняются тем, что эти подписки еще не вступили в силу.

\subsection{Выборка с сортировкой по нескольким полям}

В листинге \ref{lst:sorted.sql} представлен запрос, формирующий выборку из 5-ти сериалов, при этом сериалы отсортированы по возрастанию числа сезонов и по убыванию стоимости.

\lstinputlisting[caption={sorted.sql},label={lst:sorted.sql}]{sorted.sql}

Выборка, сформированная данным запросом, представлена на рис. \ref{fig:sorted}.

\begin{figure}[H]
	\centering
	\includegraphics[width=0.5\textwidth]{sorted.png}
	\caption{Выборка, сформированная \code{sorted.sql}}
	\label{fig:sorted}
\end{figure}

\subsection{Запрос с вычислением совокупных характеристик}

В листинге \ref{lst:aggregate.sql} представлен запрос, формирующий одну строку, содержащую общее число пользователей, максимальную длину имени, минимальную длину фамилии, средний возраст пользователей, а также число ползователей мужского пола.

\lstinputlisting[caption={aggregate.sql},label={lst:aggregate.sql}]{aggregate.sql}

Выборка, сформированная данным запросом, представлена на рис. \ref{fig:aggregate}.

\begin{figure}[H]
	\centering
	\includegraphics[width=1\textwidth]{aggregate.png}
	\caption{Выборка, сформированная \code{aggregate.sql}}
	\label{fig:aggregate}
\end{figure}

\subsection{Выборка данных из связанных таблиц}

В листинге \ref{lst:join1.sql} представлен запрос, в котором соединяются 2 таблицы --- \code{series} (сериалы) и \code{series\_translation} (переводы сериалов) для формирования названия сериала с наибольшим числом сезонов и наибольшей стоимостью.

\lstinputlisting[caption={join1.sql},label={lst:join1.sql}]{join1.sql}

Выборка, сформированная данным запросом, представлена на рис. \ref{fig:join1}.

\begin{figure}[H]
	\centering
	\includegraphics[width=0.5\textwidth]{join1.png}
	\caption{Выборка, сформированная \code{join1.sql}}
	\label{fig:join1}
\end{figure}

В листинге \ref{lst:join2.sql} представлен запрос, в котором соединяются 3 таблицы --- \code{movie\_translation} (переводы фильмов), \code{language} (языки) и \code{user\_movie} (пользователи -- фильмы) для определения наиболее часто приобретаемого фильма, который не является эпизодом сериала. Выборка состоит из 2-х записей, содержащих идентификатор фильма, имя фильма, локаль и частоту его покупки, для 2-х локалей: русской и английской.

\lstinputlisting[caption={join2.sql},label={lst:join2.sql}]{join2.sql}

Выборка, сформированная данным запросом, представлена на рис. \ref{fig:join2}.

\begin{figure}[H]
	\centering
	\includegraphics[width=0.5\textwidth]{join2.png}
	\caption{Выборка, сформированная \code{join2.sql}}
	\label{fig:join2}
\end{figure}

\subsection{Запрос с подзапросами}

В листинге \ref{lst:inner.sql} представлен запрос, который аналогично предыдущему запросу выводит название фильма, который наиболее часто покупается пользователями, но уже с помощью вложенных подзапросов.
\lstinputlisting[caption={inner.sql},label={lst:inner.sql}]{inner.sql}

Выборка, сформированная данным запросом, представлена на рис. \ref{fig:inner}.

\begin{figure}[H]
	\centering
	\includegraphics[width=0.4\textwidth]{inner.png}
	\caption{Выборка, сформированная \code{inner.sql}}
	\label{fig:inner}
\end{figure}

Выборка получилась такой же, как и при предыдущем запросе.

\subsection{Запрос с ограничением результата группировки}

В листинге \ref{lst:group.sql} представлен запрос, находящий пользователей, у которых больше 42 подписок, с помощью ограничения результата группировки по идентификатору пользователя.

\lstinputlisting[caption={group.sql},label={lst:group.sql}]{group.sql}

Выборка, сформированная данным запросом, представлена на рис. \ref{fig:group}.

\begin{figure}[H]
	\centering
	\includegraphics[width=0.2\textwidth]{group.png}
	\caption{Выборка, сформированная \code{group.sql}}
	\label{fig:group}
\end{figure}

\subsection{Добавление записей в таблицы}
%TODO

\end{document}
