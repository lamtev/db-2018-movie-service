\include{settings}

\begin{document}

\begin{titlepage}
\begin{center}
	САНКТ-ПЕТЕРБУРГСКИЙ ПОЛИТЕХНИЧЕСКИЙ УНИВЕРСИТЕТ\\ ПЕТРА ВЕЛИКОГО\\[0.3cm]
	\par\noindent\rule{10cm}{0.4pt}\\[0.3cm]
	Институт компьютерных наук и технологий \\[0.3cm]
	Кафедра компьютерных систем и программных технологий\\[4cm]
	
	Отчет по лабораторной работе № 1\\[3mm]
	Дисциплина: <<Базы данных>>\\[3mm]
	Тема: <<Разработка структуры БД>>\\[7cm]
\end{center}

\begin{flushleft}
	\hspace*{5mm} Выполнил студент гр. 43501/3  \hspace*{2.5cm}\sign[3cm]\hspace*{3.0mm} А.Ю. Ламтев\\
	\hspace*{10.4cm} (подпись)\\[3mm]
	\hspace*{5mm} Преподаватель \hspace*{6.0cm}\sign[3cm]\hspace*{2mm} А.В. Мяснов\\
	\hspace*{10.4cm} (подпись)\\[3mm]
	\hspace*{11.1cm} <<\sign[7mm]>> \sign[27mm] \the\year\hspace{1mm} г.
\end{flushleft}

\vfill

\begin{center}
	Санкт-Петербург\\
	\the\year
\end{center}
\end{titlepage}
\addtocounter{page}{1}

\tableofcontents
\newpage

\section{Цели работы}

Познакомиться с возможностями реализации более сложной обработки данных на стороне сервера с помощью хранимых процедур и триггеров.

\section{Программа работы}

\begin{enumerate}
	\item Создание двух триггеров: один триггер для автоматического заполнения ключевого поля, второй триггер для контроля целостности данных в подчиненной таблице при удалении/изменении записей в главной таблице.
	\item Создание триггера в соответствии с индивидуальным заданием, полученным у преподавателя.
	\item Создание триггера в соответствии с индивидуальным заданием, вызывающего хранимую процедуру.
\end{enumerate}
 
\section{Триггер для автоматического заполнения ключевого поля}

В листинге \ref{lst:pk-trigger.sql} представлен триггер, автоматически заполняющий поле с первичным ключом при добавлении записи в таблицу \code{language}, а также продемонстрирована его работа.

\lstinputlisting[caption={pk-trigger.sql},label={lst:pk-trigger.sql}]{pk-trigger.sql}

 
\section{Триггер для контроля целостности данных}

В листинге \ref{lst:fk-consistency-trigger.sql} представлен триггер, который позволяет поддерживать целостность данных в подчиненной таблице при удалении записей в главной таблице.

\lstinputlisting[caption={fk-consistency-trigger.sql},label={lst:fk-consistency-trigger.sql}]{fk-consistency-trigger.sql}

После удаления записи из таблицы \code{player}, первичный ключ которой является внешним ключом для таблицы \code{team}, также удаляется соответствующая запись из таблицы \code{team}.

\section{Триггер, вызывающий ХП при добавлении новой подписки пользователю}

\paragraph{Задание:} При добавлении новой подписки пользователю вызвать процедуру, которая на основании данных о подписках пользователя и о популярности фильмов по категориям формирует список предложений для заданного пользователя. Данная процедура \\\code{up\_to\_n\_movie\_recommendations\_for\_user(n SMALLINT, user\_id BIGINT)} была разработана в рамках предыдущей лабораторной работы.

\lstinputlisting[caption={new-subscription-trigger.sql},label={lst:new-subscription-trigger.sql}]{new-subscription-trigger.sql}

\section{Триггер, реализующий проверку на дубли при добавлении или изменении подписок}

\paragraph{Задание:} Реализовать проверку на дубли при добавлении или изменении подписок. В случае дубля выбрасывать исключение.

\lstinputlisting[caption={check-for-duplicates.sql},label={lst:check-for-duplicates.sql}]{check-for-duplicates.sql}

Проверка на дубли осуществляется следующим образом. При добавлении в таблицу \code{subscription\_series\_season} новой пары <<подписка -- сезон сериала>> происходит проверка всех подписок данного пользователя на наличие данного сезона сериала.

\section{Выводы}

В результате работы был изучен принцип работы триггеров, а также синтаксис, который позволяет их создавать, и затем было создано несколько триггеров, срабатывающих при добавлении, обновлении или удалении записей, удовлетворяющих определенным условиям для определенных таблиц.

\end{document}
