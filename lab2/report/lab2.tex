\include{settings}

\begin{document}

\begin{titlepage}
\begin{center}
	САНКТ-ПЕТЕРБУРГСКИЙ ПОЛИТЕХНИЧЕСКИЙ УНИВЕРСИТЕТ\\ ПЕТРА ВЕЛИКОГО\\[0.3cm]
	\par\noindent\rule{10cm}{0.4pt}\\[0.3cm]
	Институт компьютерных наук и технологий \\[0.3cm]
	Кафедра компьютерных систем и программных технологий\\[4cm]
	
	Отчет по лабораторной работе № 1\\[3mm]
	Дисциплина: <<Базы данных>>\\[3mm]
	Тема: <<Разработка структуры БД>>\\[7cm]
\end{center}

\begin{flushleft}
	\hspace*{5mm} Выполнил студент гр. 43501/3  \hspace*{2.5cm}\sign[3cm]\hspace*{3.0mm} А.Ю. Ламтев\\
	\hspace*{10.4cm} (подпись)\\[3mm]
	\hspace*{5mm} Преподаватель \hspace*{6.0cm}\sign[3cm]\hspace*{2mm} А.В. Мяснов\\
	\hspace*{10.4cm} (подпись)\\[3mm]
	\hspace*{11.1cm} <<\sign[7mm]>> \sign[27mm] \the\year\hspace{1mm} г.
\end{flushleft}

\vfill

\begin{center}
	Санкт-Петербург\\
	\the\year
\end{center}
\end{titlepage}
\addtocounter{page}{1}

\tableofcontents
\newpage

\section{Цели работы}

Познакомиться с основами проектирования схемы БД, языком описания сущностей и ограничений БД SQL-DDL.

\section{Программа работы}

\begin{enumerate}
	\item Самостоятельное изучение SQL-DDL.
	\item Создание скрипта БД в соответствии с согласованной схемой. Должны присутствовать первичные и внешние ключи, ограничения на диапазоны значений. Демонстрация скрипта преподавателю. 
	\item Создание скрипта, заполняющего все таблицы БД данными.
	\item Выполнение SQL-запросов, изменяющих схему созданной БД по заданию преподавателя. Демонстрация их работы преподавателю.
\end{enumerate}
 
%На рис. \ref{fig:movie-service-diagram} изображена \code{UML} диаграмма БД, соответствующей заданию.

%\begin{figure}[H]
%	\centering
%	\includegraphics[width=1.0\textwidth]{diagrams/movie-service-diagram}
%	\caption{\code{UML} диаграмма БД}
%	\label{fig:movie-service-diagram}
%\end{figure}

\section{Выводы}


\end{document}
